
\documentclass[11pt,a4paper]{report}%especifica o tipo de documento que tenciona escrever: carta, artigo, relatório... neste caso é um relatório
% [11pt,a4paper] Define o tamanho principal das letras do documento. caso não especifique uma delas, é assumido 10pt
% a4paper -- Define o tamanho do papel.

\usepackage[portuges]{babel}%Babel -- irá activar automaticamente as regras apropriadas de hifenização para a língua todo o
                                   %-- o texto gerado é automaticamente traduzido para Português.
                                   %  Por exemplo, “chapter” irá passar a “capítulo”, “table of contents” a “conteúdo”.
                                   % portuges -- específica para o Português.
\usepackage[utf8]{inputenc} % define o encoding usado texto fonte (input)--usual "utf8" ou "latin1

\usepackage{graphicx} %permite incluir graficos, tabelas, figuras
\usepackage{url} % para utilizar o comando \url{}
\usepackage{enumerate} %permite escolher, nas listas enumeradas, se os iems sao marcados com letras ou s-romanos em vez de numeracao normal

%\usepackage{apalike} % gerar biliografia no estilo 'named' (apalike)

\usepackage{color} % Para escrever em cores

\usepackage{multirow} %tabelas com multilinhas
\usepackage{array} %formatação especial de tabelas em array

\usepackage[pdftex]{hyperref} % transformar as referências internas do seu documento em hiper-ligações.

%Exemplos de fontes -- nao e vulgar mudar o tipo de fonte
%\usepackage{tgbonum} % Fonte de letra: TEX Gyre Bonum
%\usepackage{lmodern} % Fonte de letra: Latin Modern Sans Serif
%\usepackage{helvet}  % Fonte de letra: Helvetica
%\usepackage{charter} % Fonte de letra:Charter

\definecolor{saddlebrown}{rgb}{0.55, 0.27, 0.07} % para definir uma nova cor, neste caso 'saddlebrown'

\usepackage{listings}  

\definecolor{codegreen}{rgb}{0,0.6,0}
\definecolor{codegray}{rgb}{0.5,0.5,0.5}
\definecolor{codeblue}{rgb}{0,0,255}
\definecolor{backcolour}{rgb}{0.95,0.95,0.92}
\definecolor{codefirebrick}{rgb}{160,0,0}

\lstdefinestyle{mystyle}{
	backgroundcolor=\color{backcolour},   
	commentstyle=\color{codegreen},
	keywordstyle=\color{codefirebrick},
	numberstyle=\tiny\color{codegray},
	stringstyle=\color{codeblue},
	basicstyle=\ttfamily\footnotesize,
	breakatwhitespace=false,         
	breaklines=true,                 
	captionpos=b,                    
	keepspaces=true,                 
	numbers=left,                    
	numbersep=5pt,                  
	showspaces=false,                
	showstringspaces=false,
	showtabs=false,                  
	tabsize=2
}

\lstset{style=mystyle}

\usepackage{xspace} % deteta se a seguir a palavra tem uma palavra ou um sinal de pontuaçao se tiver uma palavra da espaço, se for um sinal de pontuaçao nao da espaço

\parindent=0pt %espaço a deixar para fazer a  indentação da primeira linha após um parágrafo
\parskip=2pt % espaço entre o parágrafo e o texto anterior

\setlength{\oddsidemargin}{-1cm} %espaço entre o texto e a margem
\setlength{\textwidth}{18cm} %Comprimento do texto na pagina
\setlength{\headsep}{-1cm} %espaço entre o texto e o cabeçalho
\setlength{\textheight}{23cm} %altura do texto na pagina

% comando '\def' usado para definir abreviatura (macros)
% o primeiro argumento é o nome do novo comando e o segundo entre chavetas é o texto original, ou sequência de controle, para que expande
\def\darius{\textsf{Darius}\xspace}
\def\antlr{\texttt{AnTLR}\xspace}
\def\pe{\emph{Publicação Eletrónica}\xspace}
\def\titulo#1{\section{#1}}    %no corpo do documento usa-se na forma '\titulo{MEU TITULO}'
\def\super#1{{\em Supervisor: #1}\\ }
\def\area#1{{\em \'{A}rea: #1}\\[0.2cm]}
\def\resumo{\underline{Resumo}:\\ }





\title{Processamento de Linguagens e Compiladores (3º Ano)\\
       \textbf{Trabalho Prático 2}\\ Relatório de Desenvolvimento
       }
\author{Bruno Fernandes\\ (a95972) \and Nelson Almeida\\ (a97610)
         \and Nuno Costa\\ (a97610)
       }
\date{15/01/2023}

\begin{document} % corpo do documento
\maketitle % apresentar titulo, autor e data


\begin{abstract}  % resumo do documento
\par O trabalho prático 2, no âmbito da UC de Processamento de Linguaguens e Compiladores alude-nos à criação de uma linguagem imperativa a nosso gosto bem como a criação de um compilador usando  os módulos de gramáticas tradutoras do Python. \\ 
  Além disso, esta gramática tem de ser capaz de gerar código assembly a partir da linguagem imperativa, com recurso a algumas ferramentas como o lex e o yacc do Python. \\
 Assim sendo, durante o realizar do relatório tentaremos sempre explicar de uma forma clara e suncinta todas as decisões tomadas pos nós,bem como as produções implementadas na gramática e ainda como foi desenvolvido o compilador.
\end{abstract}


\tableofcontents


\chapter{Introdução} \label{chap:intro} %referência cruzada
No âmbito da disciplina de Processamento de Linguagens e Compiladores foi-nos proposto pelo docente Pedro Rangel Henriques um trabalho de grupo cujos objetivos principais são: tornar-nos capazes de escrever gramáticas com uma maior facilidade, sermos capazes de desenvolver um processador de linguagens a partir de uma gramática tradutora e ainda de desenvolver um complilador produzindo código para uma máquina de stack virtual.
\\
\\
A linguagem a ser usada na realização deste projeto, será uma linguagem imperativa simples com regras definidas pelo grupo. 
\\
O compilador desenvolvido para a nossa linguagem terá de gerar pseudo-código Assembly para uma VM, com base na gramática independente do contexto(GIC) que definimos.
\\
\\
Neste documento apresentamos a nossa resolução para cada um dos problemas propostos, com recurso aos módulos 'Yacc/Lex' do 'PLY/Python'.


\newpage


\section*{Estrutura do Relatório}
O relatório está organizado da seguinte forma:\\
Começamos por fazer uma pequena introdução, capítulo 1, onde referimos o objetivo do trabalho a desenvolver. \\
No capítulo 2 aprensentamos o enunciado dos problema proposto.\\
O capítulo 3 demonstramos como está organizado o nosso trabalho.\\
No capítulo seguinte, demonstramos o funcionamento de vários testes realizados pelo grupo.\\
E, por fim, o ultimo capítulo contém a conclusão do trabalho realizado.

\newpage



\chapter{Problema Proposto}

Pretende-se que comece por definir uma linguagem de programação imperativa simples, a seu gosto.
Apenas deve ter em consideração que essa linguagem terá de permitir:
\begin{itemize}
	\item declarar variáveis at ómicas do tipo inteiro, com os quais se podem realizar as habituais operações aritméticas, relacionais e lógicas.
	\item efetuar instruções algorítmicas básicas como a atribuição do valor de expressões numéricas a variáveis.
	\item ler do standard input e escrever no standard output.
    \item efetuar instruções de seleção para o controlo do fluxo de execução.
    \item efetuar instruções de repetição(cíclicas) para o controlo de fluxo de execução, permitindo o seu aninhamento.\\
    \underline{Note} que deve implementar pelo menos o ciclo \textbf{while-do},\textbf{repeat-until} ou \textbf{for-do}.
\end{itemize}
\\ \\
Adicionalmente deve ainda suportar,à sua escolha, uma das duas funcionalidades seguintes:
\begin{itemize}
    \item declarar e manusear variáveis estruturadas do tipo array( a 1 ou 2 dimensões) de inteiros, em relação aos quais é apenas permitida a operação de indexação(índice inteiro).
    \item definir e invocar subprogramas sem parâmetros mas que possam retornar um resultado do tipo inteiro.
\end{itemize}

\newpage

\chapter{Concepção da Resolução}
\section{Organização e estrutura}
O nosso trabalho pode ser divido em \textbf{4} partes:
\begin{itemize}
\item Cosntrução da \textbf{GIC} que define a estrutura sintática da nossa liguaguem.
\item Construção do analizador léxico, \textbf{lexer} .
\item Construção do analizador sintático, \textbf{parser} .
\item Conversão das instruções para código \textbf{Assembly} da  VM.    
\end{itemize}
Todas as funcionalidades descritas neste capítulo podem ser encontradas no anexo A do documento. \\ \\

\section{GIC}
A nossa linguagem é gerada pela seguinte grámatica independente de contexto:
\begin{verbatim}
Programa : Decls 
         | Corpo
         | Decls Corpo

Decls    : Decl
         | Decl Decls

Decl     : VAR ID 
         | LISTA ID
         | LISTA ID INT 
         | LISTA ID COM lista
         | MATRIZ ID 
         | MATRIZ ID INT INT

Atrib    : VAR ID COM expr
         | ALTERNA ID ABREPR expr FECHAPR COM expr
         | ALTERNA ID COM expr
         | VAR ID COM entradas 
         | ALTERNA ID COM entradas

Corpo    : Proc
         | Proc Corpo
         | Atrib 
         | Atrib Corpo

expr     : INT 
         | ID 
         | exprArit
         | exprRel

lista    : ABREPR elems FECHAPR

elems    : INT  
         | INT VIRG elems 

exprArit : expr SOMA expr
         | expr MENUS expr
         | expr SOMANBEZES expr
         | expr DIBIDE expr
         | expr SOBRAS expr

exprRel  : NOUM ABREPC expr FECHAPC
         | GEMEO ABREPC expr VIRG expr FECHAPC
         | NAOGEMEO ABREPC expr VIRG expr FECHAPC
         | MAISPIQUENO ABREPC expr VIRG expr FECHAPC
         | MAISPIQUENOOUGEMEO ABREPC expr VIRG expr FECHAPC
         | MAISGRANDE ABREPC expr VIRG expr FECHAPC
         | MAISGRANDEOUGEMEO ABREPC expr VIRG expr FECHAPC
         | expr IE expr
         | expr OUE expr

Proc     : if
         | while 
         | saidas
        
if       : SE exprRel LOGO Corpo FIM
         | SE exprRel LOGO Corpo SENAO Corpo FIM

while    : ENQUANTO exprRel FAZ Corpo FIM

saidas   : SAIDAS ID 
         | SAIDAS expr            
\end{verbatim}

\newpage

\section{Lexer}
O analisador léxico, \textbf{lexer}, é o responsável por 'capturar' os simbolos terminais(\textit{tokens}) da nossa linguagem através de expressões regulares. Para a implementação do analisador léxico utilizamos o módulo 'Lex' do 'PLY/Python'.\\ \\
Os \textit{tokens} e respetivas expressões regulares da nossa linguagem são os seguintes:

\begin{verbatim}

ABRECHAV : '\{'
FECHACHAV : '\}'
ABREPC : '\('
FECHAPC : '\)'
ABREPR : '\['
FECHAPR : '\]'
VIRG : '\,'
SOMA : '\+'
MENUS : '\-'
SOMANBEZES : '\*'
DIBIDE : '\/'
SOBRAS : '\%'
STRING : '\"w+\"|\'w+\''
ID : r'\w+'
INT : '\d+'
VAR : 'var'
COM : 'com'
MAISGRANDE : 'maisGrande'
MAISPIQUENO : 'maisPiqueno'
GEMEO : 'gemeo'
NAOGEMEO : 'naogemeo'
MAISGRANDEOUGEMEO : 'maisGrandeOuGemeo'
MAISPIQUENOOUGEMEO : 'maisPiquenoOuGemeo'
IE : 'ie'
OUE : 'oue'
NOUM : 'noum'
ALTERNA : 'alterna'
LISTA : 'lista'
MATRIZ : 'matriz'
BUSCA : 'busca'
SWAP : 'swap'
SENAO : 'senao'
SE : 'se'
ENTAO : 'entao'
FIM : 'fim'
ENQUANTO : 'enquanto'
FAZ : 'faz'
ENTRADAS : 'entradas'
\end{verbatim}

A implementação do analisador léxico pode ser encontrada no anexo A do documento. 
\\ \\


\section{Parser e geração do código Assembly da VM}
O analisador sintático, \textbf{parser}, é  o responsável por verificar se o código escrito na nossa linguagem está correto sintaticamente, isto é, se  o código respeita as regras gramaticais definidas. \\ \\

No caso de não existirem erros sintáticos o \textbf{parser} converte o código da nossa linguagem em código \textbf{Assemby} da máquina virtual. caso existam erros, então será mostrado ao utilizador uma mensagem do erro sintático produzido. \\ \\
A implementação do analisador sintático pode ser encontrada no anexo A do nosso relatório.


\subsection{Algumas notas sobre declaração de variáveis}

Na geração do código para declarar uma variavel sem valor fazemos:
\begin{lstlisting}
PUSHI 0    
\end{lstlisting}
Sendo a variável predefinida a 0.\\

Para declarar uma variavel com valor fazemos:
\begin{lstlisting}
PUSHI <valor>
STOREG <endereco>
\end{lstlisting}
\\
Para declarar uma lista de valores temos de fazer sempre:
\begin{lstlisting}
PUSHN <tamanho>
\end{lstlisting}
Inicializando todos os valores da lista a 0.\\ \\
Seja a uma lista de tamanho 3, por exemplo:
\begin{lstlisting}
lista a tamanho
lista a com [valor1,valor2,valor3]
\end{lstlisting}
Para atribuir valores à lista fazemos:
\begin{lstlisting}
PUSHN <tamanho>
PUSHGP
PUSHI 0
PUSHI <valor1>
STOREN
PUSHGP
PUSHI 1
PUSHI <valor2>
STOREN
PUSHGP
PUSHI 2
PUSHI <valor3>
STOREN 
\end{lstlisting}


No caso das matrizes, para declarar uma matriz fazemos: 
\begin{lstlisting}
matriz <nomeDaMatriz> <tamanho1> <tamanho2>    
\end{lstlisting}
Caso a matriz m seja de tamanho 2x2, então gera-se:
\begin{lstlisting}
PISHN 4
\end{lstlisting}
Inicializando todos os valores a 0. \\ \\

Para alterar os valores de uma matriz temos duas maneiras de o fazer:
\begin{enumerate}
\item Alteramos uma posição em especifico:
\begin{lstlisting}
alterna <nomeDaMatriz> [<indice1>] [<indice2>]  com <valor>    
\end{lstlisting}
\item Alteramos uma linha da matriz passando uma lista:
\begin{lstlisting}
alterna <nomeDaMatriz> [<indice1>] com [valor1,valor2]
\end{lstlisting}
\end{enumerate}

Tomando como exemplo uma matriz 2x2, ao fazer a alteração de valores pela 1ª opção:
\begin{lstlisting}
alterna m [0][1] com <valor>
\end{lstlisting}
Gera-se o seguinte:
\begin{lstlisting}
PUSHN 4
PUSHGP
PUSHI 0
PADD
PUSHI 0
PUSHI <valor>
STOREN
\end{lstlisting}

Fazendo o mesmo mas de acordo com a 2ª opção:
\begin{lstlisting}
alterna m [0] com [valor1,valor2]
\end{lstlisting}
Gera-se o seguinte:
\begin{lstlisting}
PUSHN 4
PUSHGP
PUSHI 0
PADD
PUSHI 0
PUSHI 2
MUL
PADD
PUSHI 0
PUSHI <valor1>
STOREN
PUSHGP
PUSHI 0
PADD
PUSHI 0
PUSHI 2
MUL
PADD
PUSHI 1
PUSHI <valor2>
STOREN
\end{lstlisting}

\chapter{Demonstração do Funcionamento}
\section{Geração e  execução de código Assembly}
Para utilizar a nossa linguagem, o utilizador tem 3 opções:
\begin{enumerate}
\item Escrever instruções de acordo com as regras gramaticais da linguagem.
\begin{verbatim} 
>> python3 yacc.py
\end{verbatim}
\item Escrever e guardar as instruções num ficheiro .plo de acordo com as regras gramaticais da linguagem.
\begin{verbatim} 
>> python3 yacc.py <ficheiro de input> 
\end{verbatim}
\item Escrever e guardar as instruções num ficheiro .plo de acordo com as regras gramaticais da linguagem e escolher o ficheiro de saída.
\begin{verbatim} 
>> python3 yacc.py <ficheiro de input> <ficheiro de output>
\end{verbatim} 

Por exemplo:
\begin{verbatim} 
>> python3 yacc.py .\testes\factorial.plo output.vm
\end{verbatim} 

\underline{Nota:} Caso o utilizador escolha fazer a opção 1 ou 2 é criado um ficheiro "a.vm" onde será guardado o código Assembly gerado.
\end{enumerate}





\section{Teste 1}
Calcula o fatorial de um número passado como input. \\
\underline{Ficheiro de input:} 'factorial.plo'
\subsection{Conteúdo do ficheiro}

\begin{lstlisting}[language=python]
saidas "Factorial: "
var n com entradas
saidas n
var res com 1

enquanto (maisGrande(n,0)) faz {
    alterna res com res * n
    alterna n com n - 1
} fim

saidas "\nResultado: "
saidas res
\end{lstlisting}

\subsection{Código assembly gerado}
\begin{lstlisting}[languague=Assembler]
START
PUSHS "Factorial: "
WRITES
READ
ATOI
STOREG 0
PUSHG 0
WRITEI
PUSHI 1
STOREG 1
l0c: NOP
PUSHG 0
PUSHI 0
SUP
JZ l0f
PUSHG 1
PUSHG 0
MUL
STOREG 1
PUSHG 0
PUSHI 1
SUB
STOREG 0
JUMP l0c
l0f: NOP
PUSHS "\nResultado: "
WRITES
PUSHG 1
WRITEI
STOP
\end{lstlisting}


\subsection{Execução da VM com o código gerado}
\begin{lstlisting}[language=python]
Factorial: 5
Resultado: 120
\end{lstlisting}


\section{Teste 2}
Procura determinado número pelo seu índice. \\
\underline{Ficheiro de input:} 'busca\_no\_array.plo'.

\subsection{Conteúdo do ficheiro}

\begin{lstlisting}[language=python]
lista a 10
lista a com [1,2,3,4,5,6,7,8,9,10]

saidas "Introduza um indice do array:\n"
var i com entradas

var x com busca a[i]

saidas "Valor: "
saidas x
\end{lstlisting}

\subsection{Código assembly gerado}
\begin{lstlisting}[language=Assembler]
PUSHN 10
START
PUSHGP
PUSHI 0
PUSHI 1
STOREN
PUSHGP
PUSHI 1
PUSHI 2
STOREN
PUSHGP
PUSHI 2
PUSHI 3
STOREN
PUSHGP
PUSHI 3
PUSHI 4
STOREN
PUSHGP
PUSHI 4
PUSHI 5
STOREN
PUSHGP
PUSHI 5
PUSHI 6
STOREN
PUSHGP
PUSHI 6
PUSHI 7
STOREN
PUSHGP
PUSHI 7
PUSHI 8
STOREN
PUSHGP
PUSHI 8
PUSHI 9
STOREN
PUSHGP
PUSHI 9
PUSHI 10
STOREN
PUSHS "Introduza um indice do array:\n"
WRITES
READ
ATOI
STOREG 10
PUSHGP
PUSHI 0
PADD
PUSHG 10
LOADN
STOREG 11
PUSHS "Valor: "
WRITES
PUSHG 11
WRITEI
STOP
\end{lstlisting}

\subsection{Código gerado pela VM}
\begin{lstlisting}[language=python]
Introduza um indice do array: 3
Valor: 4
\end{lstlisting}


\section{Teste 3}
Lê os 5 valores de um array passados como input. \\
\underline{Ficheiro de input:} 'read\_array.plo'.

\subsection{Conteúdo do ficheiro}

\begin{lstlisting}[language=python]
var n com 5
var i com 0
lista a 5

enquanto (maisPiqueno(i,n)) faz {
    alterna a [i] com entradas
    alterna i com i + 1
} fim

saidas "Array gerado:\n"
saidas a
\end{lstlisting}

\subsection{Código assembly gerado}
\begin{lstlisting}[languague=Assembler]
PUSHI 5
STOREG 0
START
PUSHI 0
STOREG 1
PUSHN 5
l0c: NOP
PUSHG 1
PUSHG 0
INF
JZ l0f
PUSHGP
PUSHI 2
PADD
PUSHG 1
READ
ATOI
STOREN
PUSHG 1
PUSHI 1
ADD
STOREG 1
JUMP l0c
l0f: NOP
PUSHS "Array gerado:\n"
WRITES
PUSHS "["
WRITES
PUSHGP
PUSHI 2
PADD
PUSHI 0
LOADN
WRITEI
PUSHS ","
WRITES
PUSHGP
PUSHI 2
PADD
PUSHI 1
LOADN
WRITEI
PUSHS ","
WRITES
PUSHGP
PUSHI 2
PADD
PUSHI 2
LOADN
WRITEI
PUSHS ","
WRITES
PUSHGP
PUSHI 2
PADD
PUSHI 3
LOADN
WRITEI
PUSHS ","
WRITES
PUSHGP
PUSHI 2
PADD
PUSHI 4
LOADN
WRITEI
PUSHS "]"
WRITES
STOP
\end{lstlisting}

\subsection{Código gerado pela VM}
\begin{lstlisting}[language=python]
Array gerado:
[1,2,3,4,5]
\end{lstlisting}

\section{Teste 4}
Realiza o produto de vários números passados como input. \\
\underline{Ficheiro de input:} 'produtorio.plo'.

\subsection{Conteúdo do ficheiro}

\begin{lstlisting}[language=python]
saidas "Quantos numeros? "
var n com entradas
saidas n
var res com 1
var x com 1

enquanto (maisGrande(n,0)) faz {
    alterna x com entradas
    alterna res com res * x
    alterna n com n - 1
} fim

saidas "\nResultado: "
saidas res
\end{lstlisting}


\subsection{Código assembly gerado}
\begin{lstlisting}[languague=Assembler]
START
PUSHS "Quantos numeros? "
WRITES
READ
ATOI
STOREG 0
PUSHG 0
WRITEI
PUSHI 1
STOREG 1
PUSHI 1
STOREG 2
l0c: NOP
PUSHG 0
PUSHI 0
SUP
JZ l0f
READ
ATOI
STOREG 2
PUSHG 1
PUSHG 2
MUL
STOREG 1
PUSHG 0
PUSHI 1
SUB
STOREG 0
JUMP l0c
l0f: NOP
PUSHS "\nResultado: "
WRITES
PUSHG 1
WRITEI
STOP   
\end{lstlisting}

\subsection{Código gerado pela VM}
\begin{lstlisting}[language=python]
Quantos numeros? 5
Resultado: 120
\end{lstlisting}

\section{Teste 5}
A partir de 3 arrays de tamanho 3, crai uma matriz de tamanho 3x3. \\
\underline{Ficheiro de input:} 'matriz.plo'.

\subsection{Conteúdo do ficheiro}

\begin{lstlisting}[language=python]
matriz m 3 3

alterna m [0] com [1,2,3]
alterna m [1] com [4,5,6]
alterna m [2] com [7,8,9]

saidas m
\end{lstlisting}

\subsection{Código assembly gerado}
\begin{lstlisting}[languague=Assembler]
PUSHN 9
START
PUSHGP
PUSHI 0
PADD
PUSHI 0
PUSHI 3
MUL
PADD
PUSHI 0
PUSHI 1
STOREN
PUSHGP
PUSHI 0
PADD
PUSHI 0
PUSHI 3
MUL
PADD
PUSHI 1
PUSHI 2
STOREN
PUSHGP
PUSHI 0
PADD
PUSHI 0
PUSHI 3
MUL
PADD
PUSHI 2
PUSHI 3
STOREN
PUSHGP
PUSHI 0
PADD
PUSHI 1
PUSHI 3
MUL
PADD
PUSHI 0
PUSHI 4
STOREN
PUSHGP
PUSHI 0
PADD
PUSHI 1
PUSHI 3
MUL
PADD
PUSHI 1
PUSHI 5
STOREN
PUSHGP
PUSHI 0
PADD
PUSHI 1
PUSHI 3
MUL
PADD
PUSHI 2
PUSHI 6
STOREN
PUSHGP
PUSHI 0
PADD
PUSHI 2
PUSHI 3
MUL
PADD
PUSHI 0
PUSHI 7
STOREN
PUSHGP
PUSHI 0
PADD
PUSHI 2
PUSHI 3
MUL
PADD
PUSHI 1
PUSHI 8
STOREN
PUSHGP
PUSHI 0
PADD
PUSHI 2
PUSHI 3
MUL
PADD
PUSHI 2
PUSHI 9
STOREN
PUSHS "["
WRITES
PUSHS "["
WRITES
PUSHGP
PUSHI 0
PADD
PUSHGP
PUSHI 0
PUSHI 3
MUL
PADD
PUSHI 0
LOADN
WRITEI
POP 1
PUSHS ","
WRITES
PUSHGP
PUSHI 0
PADD
PUSHGP
PUSHI 0
PUSHI 3
MUL
PADD
PUSHI 1
LOADN
WRITEI
POP 1
PUSHS ","
WRITES
PUSHGP
PUSHI 0
PADD
PUSHGP
PUSHI 0
PUSHI 3
MUL
PADD
PUSHI 2
LOADN
WRITEI
POP 1
PUSHS "]"
WRITES
PUSHS ","
WRITES
PUSHS "["
WRITES
PUSHGP
PUSHI 0
PADD
PUSHGP
PUSHI 1
PUSHI 3
MUL
PADD
PUSHI 0
LOADN
WRITEI
POP 1
PUSHS ","
WRITES
PUSHGP
PUSHI 0
PADD
PUSHGP
PUSHI 1
PUSHI 3
MUL
PADD
PUSHI 1
LOADN
WRITEI
POP 1
PUSHS ","
WRITES
PUSHGP
PUSHI 0
PADD
PUSHGP
PUSHI 1
PUSHI 3
MUL
PADD
PUSHI 2
LOADN
WRITEI
POP 1
PUSHS "]"
WRITES
PUSHS ","
WRITES
PUSHS "["
WRITES
PUSHGP
PUSHI 0
PADD
PUSHGP
PUSHI 2
PUSHI 3
MUL
PADD
PUSHI 0
LOADN
WRITEI
POP 1
PUSHS ","
WRITES
PUSHGP
PUSHI 0
PADD
PUSHGP
PUSHI 2
PUSHI 3
MUL
PADD
PUSHI 1
LOADN
WRITEI
POP 1
PUSHS ","
WRITES
PUSHGP
PUSHI 0
PADD
PUSHGP
PUSHI 2
PUSHI 3
MUL
PADD
PUSHI 2
LOADN
WRITEI
POP 1
PUSHS "]"
WRITES
PUSHS "]"
WRITES
STOP
\end{lstlisting}

\subsection{Execução da VM com o código gerado}
\begin{lstlisting}[language=python]
[[1,2,3],[4,5,6],[7,8,9]]
\end{lstlisting}

\section{Teste 6}
Troca a posição de um certo valor do array por um outro, tendo em conta os índices . \\
\underline{Ficheiro de input:} 'swap\_array.plo'.

\subsection{Conteúdo do ficheiro}

\begin{lstlisting}[language=python]
lista a 5
lista a com [1,2,3,4,5]

saidas "Array inicial:\n"
saidas a

saidas "\nTroca do indice 1 com indice 3."

swap a [1] com [3]

saidas "Array inicial:\n"
saidas a
\end{lstlisting}

\subsection{Código assembly gerado}
\begin{lstlisting}[languague=Assembler]
PUSHN 5
START
PUSHGP
PUSHI 0
PUSHI 1
STOREN
PUSHGP
PUSHI 1
PUSHI 2
STOREN
PUSHGP
PUSHI 2
PUSHI 3
STOREN
PUSHGP
PUSHI 3
PUSHI 4
STOREN
PUSHGP
PUSHI 4
PUSHI 5
STOREN
PUSHS "Array inicial:\n"
WRITES
PUSHS "["
WRITES
PUSHGP
PUSHI 0
PADD
PUSHI 0
LOADN
WRITEI
PUSHS ","
WRITES
PUSHGP
PUSHI 0
PADD
PUSHI 1
LOADN
WRITEI
PUSHS ","
WRITES
PUSHGP
PUSHI 0
PADD
PUSHI 2
LOADN
WRITEI
PUSHS ","
WRITES
PUSHGP
PUSHI 0
PADD
PUSHI 3
LOADN
WRITEI
PUSHS ","
WRITES
PUSHGP
PUSHI 0
PADD
PUSHI 4
LOADN
WRITEI
PUSHS "]"
WRITES
PUSHS "\nTroca do indice 1 com indice 3."
WRITES
PUSHG 1
PUSHG 3
STOREG 1
STOREG 3
PUSHS "Array inicial:\n"
WRITES
PUSHS "["
WRITES
PUSHGP
PUSHI 0
PADD
PUSHI 0
LOADN
WRITEI
PUSHS ","
WRITES
PUSHGP
PUSHI 0
PADD
PUSHI 1
LOADN
WRITEI
PUSHS ","
WRITES
PUSHGP
PUSHI 0
PADD
PUSHI 2
LOADN
WRITEI
PUSHS ","
WRITES
PUSHGP
PUSHI 0
PADD
PUSHI 3
LOADN
WRITEI
PUSHS ","
WRITES
PUSHGP
PUSHI 0
PADD
PUSHI 4
LOADN
WRITEI
PUSHS "]"
WRITES
STOP
\end{lstlisting}


\subsection{Código gerado pela VM}
\begin{lstlisting}[language=python]
Array inicial:
[1,2,3,4,5]
Troca do indice 1 com indice 3.Array inicial:
[1,4,3,2,5]
\end{lstlisting}





\chapter{Conclusão}
No decorrer deste trabalho, tentamos sempre aplicar todo e qualquer conhecimento adquirido em aulas, o que nos permitiu aprofundar e consolidar melhor a matéria lecionada nesta UC.

Consideramos que, no geral,conseguimos alcançar os objetivos esperados e desta forma temos mais bagagem no que toca à escrita de gramaticas e no desenvolvimento de compiladores de linguagens.
Este trabalho levou-nos também a obter um maior conhecimento no que diz respeito à máquina virtuale a uma maior entendimento da escrita em Assembly.

Em suma, todo o trabalho aplicado na realização deste projeto foi bastante útil para consolidar as nossas bases e dar-nos também alguma naturalidade na abordagem de certas temáticas da UC que poderão vir a ser necessárias no nosso futuro profissional.

\appendix % apendice
\chapter{Código do Programa}

\textbf{Ficheiro lex2.py}
\begin{scriptsize}
\begin{lstlisting}[language=python]
import ply.lex as lex

tokens = [
    "ASPA",
    "ID",
    "VAR",
    "COM",

    "ABREPC",
    "FECHAPC",
    "ABREPR",
    "FECHAPR",
    "ABRECHAV",
    "FECHACHAV",
    "VIRG",

    "INT",

    'SOMA',
    'MENUS',
    'SOMANBEZES',
    'DIBIDE',
    'SOBRAS',

    'MAISGRANDE',
    'MAISPIQUENO',
    'GEMEO',
    'NAOGEMEO',
    'MAISGRANDEOUGEMEO',
    'MAISPIQUENOOUGEMEO',

    'IE',
    'OUE',
    'NOUM',

    "ALTERNA",

    "LISTA",
    "MATRIZ",
    "BUSCA",
    "SWAP",

    "SENAO",
    "SE",
    "ENTAO",
    "FIM",

    "ENQUANTO",
    "FAZ",

    "ENTRADAS",
    "SAIDAS"
]

t_ignore = ' \r\n\t'

t_ABRECHAV = r"\{"
t_FECHACHAV = r"\}"
t_ABREPC = r'\('
t_FECHAPC = r'\)'
t_ABREPR = r'\['
t_FECHAPR = r'\]'
t_VIRG = r'\,'
t_SOMA = r'\+'
t_MENUS = r'\-'
t_SOMANBEZES = r'\*'
t_DIBIDE = r'\/'
t_SOBRAS = r'\%'


t_ASPA= r'\".*\"'
t_ID = r"\w+"

def t_INT(t):
    r'\d+'
    t.type = "INT"
    return t


def t_COMENTARIO(t):
    r'comentario'
    t.type = "COMENTARIO"
    return t


def t_VAR(t):
    r'var'
    t.type = "VAR"
    return t


def t_COM(t):
    r'com'
    t.type = "COM"
    return t


def t_ALTERNA(t):
    r'alterna'
    t.type = "ALTERNA"
    return t


def t_MAISGRANDE(t):
    r"maisGrande"
    t.type = "MAISGRANDE"
    return t


def t_MAISPIQUENO(t):
    r"maisPiqueno"
    t.type = "MAISPIQUENO"
    return t


def t_NAOGEMEO(t):
    r"naogemeo"
    t.type = "NAOGEMEO"
    return t


def t_GEMEO(t):
    r"gemeo"
    t.type = "GEMEO"
    return t


def t_MAISGRANDEOUGEMEO(t):
    r"maisGrandeOuGemeo"
    t.type = "MAISGRANDEOUGEMEO"
    return t


def t_MAISPIQUENOOUGEMEO(t):
    r"maisPiquenoOuGemeo"
    t.type = "MAISPIQUENOOUGEMEO"
    return t


def t_IE(t):
    r"ie"
    t.type = "IE"
    return t


def t_OUE(t):
    r"oue"
    t.type = "OUE"
    return t


def t_NOUM(t):
    r"noum"
    t.type = "NOUM"
    return t


def t_LISTA(t):
    r'lista'
    t.type = "LISTA"
    return t


def t_MATRIZ(t):
    r'matriz'
    t.type = "MATRIZ"
    return t


def t_BUSCA(t):
    r'busca'
    t.type = "BUSCA"
    return t


def t_SWAP(t):
    r'swap'
    t.type = "SWAP"
    return t


def t_SENAO(t):
    r'senao'
    t.type = "SENAO"
    return t


def t_SE(t):
    r'se'
    t.type = "SE"
    return t


def t_ENTAO(t):
    r'entao'
    t.type = "ENTAO"
    return t


def t_ENQUANTO(t):
    r'enquanto'
    t.type = "ENQUANTO"
    return t


def t_FAZ(t):
    r'faz'
    t.type = "FAZ"
    return t


def t_FIM(t):
    r'fim'
    t.type = "FIM"
    return t


def t_ENTRADAS(t):
    r"entradas"
    t.type = "ENTRADAS"
    return t


def t_SAIDAS(t):
    r"saidas"
    t.type = "SAIDAS"
    return t


def t_error(t):
    print('Illegal character: ' + t.value[0])
    t.lexer.skip(1)
    return


lexer = lex.lex()
\end{lstlisting}
\end{scriptsize}

\\ \\  \textbf{Ficheiro yacc3.py}
\begin{scriptsize} 
\begin{lstlisting}[language=python]
import ply.yacc as yacc
import random as rd
import sys
import os
import difflib
import random as rd

from lex import *


def p_Programa_Empty(p):
    '''
    Programa : Decls
             | Atrib
    '''
    parser.assembly = f'{p[1]}'


def p_Programa(p):
    '''
    Programa : Decls Corpo
             | Atrib Corpo
    '''
    parser.assembly = f'{p[1]}START\n{p[2]}STOP\n'


def p_Programa_Corpo(p):
    '''
    Programa : Corpo
    '''
    parser.assembly = f"START\n{p[1]}STOP\n"


def p_Corpo(p):
    '''
    Corpo : Codigo
    '''
    p[0] = f"{p[1]}"


def p_Codigo_Rec(p):
    '''
    Codigo : Proc Codigo
           | Atrib Codigo
    '''
    p[0] = f"{p[1]}{p[2]}"


def p_Codigo(p):
    '''
    Codigo : Proc
           | Atrib
    '''
    p[0] = f"{p[1]}"


def p_Decls(p):
    "Decls : Decl"
    p[0] = f'{p[1]}'


def p_DeclsRec(p):
    "Decls : Decl Decls"
    p[0] = f'{p[1]}{p[2]}'


def p_expr_arit(p):
    '''
    expr : exprArit
         | exprRel
    '''
    p[0] = p[1]


def p_Proc(p):
    '''
    Proc : if
         | while
         | saidas
    '''
    p[0] = p[1]


# Declaração de uma variavel sem valor
def p_Decl(p):
    "Decl : VAR ID"
    varName = p[2]
    if varName not in parser.variaveis:
        parser.variaveis[varName] = (parser.stackPointer, None)
        p[0] = "PUSHI 0\n"
        parser.stackPointer += 1
    else:
        parser.exito = False
        parser.error = f"Variável com o nome {varName} já existe"
    parser.linhaDeCodigo +=1


# Declaração de uma variável com atribuição de um valor
def p_Atrib_expr(p):
    "Atrib : VAR ID COM expr"
    varName = p[2]
    if varName not in parser.variaveis:
        value = p[4]
        parser.variaveis[varName] = (parser.stackPointer, None)
        p[0] = f"{value}STOREG {parser.stackPointer}\n"
        parser.stackPointer += 1
    else:
        parser.exito = False
        parser.error = f"Variável com o nome {varName} já existe"
    parser.linhaDeCodigo +=1


# Altera valor de um variável
def p_alterna_var(p):
    "Atrib : ALTERNA ID COM expr"
    varName = p[2]
    if varName in parser.variaveis:
        p[0] = f"{p[4]}STOREG {parser.variaveis[varName][0]}\n"
    parser.linhaDeCodigo +=1


def p_expr(p):
    "expr : INT"
    p[0] = f"PUSHI {int(p[1])}\n"


def p_expr_var(p):
    "expr : ID"
    varName = p[1]
    if varName in parser.variaveis:
        p[0] = f"PUSHG {parser.variaveis[varName][0]}\n"


def p_expr_entradas(p):
    "expr : ENTRADAS"
    p[0] = f"READ\nATOI\n"


# Declara lista sem tamanho
def p_Decl_Lista_NoSize(p):
    "Atrib : LISTA ID"
    listName = p[2]
    if listName not in parser.variaveis:
        parser.variaveis[listName] = (parser.stackPointer, 0)
        p[0] = f"PUSHN 0\n"
        parser.stackPointer += 1
    else:
        parser.error = (
            f"Variável com o nome {listName} já definida anteriormente.")
        parser.exito = False
    parser.linhaDeCodigo +=1


# Declara lista com tamanho INT
def p_DeclLista_Size(p):
    "Atrib : LISTA ID INT"
    listName = p[2]
    size = int(p[3])
    if listName not in parser.variaveis:
        if size > 0:
            parser.variaveis[listName] = (parser.stackPointer, size)
            p[0] = f"PUSHN {size}\n"
            parser.stackPointer += size
        else:
            parser.error = f"Impossível declarar um array de tamanho {size}"
            parser.exito = False
    else:
        parser.error = (
            f"Variável com o nome {listName} já definida anteriormente.")
        parser.exito = False
    parser.linhaDeCodigo +=1


# Atribui valores à lista com outra lista
def p_AtribLista_lista(p):
    "Atrib : LISTA ID COM lista"
    lista = p[4]
    varName = p[2]
    print(lista)
    if varName in parser.variaveis:
        if len(lista) == parser.variaveis[varName][1]:
            assm = ""
            for i in range(len(lista)):
                assm += f"PUSHGP\nPUSHI {parser.variaveis[varName][0]+i}\nPUSHI {int(lista[i])}\nSTOREN\n"
            p[0] = assm
        else:
            print("stackOverflow")
    else:
        parser.error = f"Variável com o nome {varName} não definida"
        parser.exito = False
    parser.linhaDeCodigo +=1


# Altera valor de um indice da lista
def p_AlternaLista_elem(p):
    "Atrib : ALTERNA ID ABREPR expr FECHAPR COM expr"
    varName = p[2]
    pos = p[4]
    if varName in parser.variaveis:
        p[0] = f"PUSHGP\nPUSHI {parser.variaveis[varName][0]}\nPADD\n{p[4]}{p[7]}STOREN\n"
    else:
        parser.error = f"Variável com o nome {varName} não definida"
        parser.exito = False
    parser.linhaDeCodigo +=1


# Declara lista sem tamanho
def p_Decl_Matriz_NoSize(p):
    "Decl : MATRIZ ID"
    listName = p[2]
    if listName not in parser.variaveis:
        parser.variaveis[listName] = (parser.stackPointer, 0, 0)
        p[0] = f"PUSHN 0\n"
        parser.stackPointer += 1
    else:
        parser.error = (
            f"Variável com o nome {listName} já definida anteriormente.")
        parser.exito = False
    parser.linhaDeCodigo +=1


# Declara matriz com tamanho INT INT
def p_DeclMatriz_Size(p):
    "Decl : MATRIZ ID INT INT"
    listName = p[2]
    size = int(p[3])
    size1 = int(p[4])
    if listName not in parser.variaveis:
        parser.variaveis[listName] = (parser.stackPointer, size, size1)
        p[0] = f"PUSHN {size*size1}\n"
        parser.stackPointer += size*size1
    else:
        parser.error = (
            f"Variável com o nome {listName} já definida anteriormente.")
        parser.exito = False
    parser.linhaDeCodigo +=1


# Função que altera o valor de um indice da matriz por outro
def p_AtribMatriz_comExpr(p):
    "Atrib : ALTERNA ID ABREPR expr FECHAPR ABREPR expr FECHAPR COM expr"
    matName = p[2]
    indice1 = p[4]
    indice2 = p[7]
    valor = p[10]
    if matName in parser.variaveis:
        if len(parser.variaveis[matName]) == 3:
            p[0] = f"PUSHGP\nPUSHI {parser.variaveis[matName][0]}\nPADD\n{indice1}PUSHI {parser.variaveis[matName][2]}\nMUL\nPADD\n{indice2}{valor}STOREN\n"
        else:
            parser.error = f"Operação inválida, variável {matName} não é uma matriz"
            parser.exito = False
    else:
        parser.error = f"Variável não declarada anteriormente"
        parser.exito = False
    parser.linhaDeCodigo +=1


# Função que altera uma lista da matriz por outra
def p_AtribMatriz_comLista(p):
    "Atrib : ALTERNA ID ABREPR expr FECHAPR COM lista"
    matName = p[2]
    if matName in parser.variaveis:
        if len(parser.variaveis[matName]) == 3:
            if len(p[7]) <= parser.variaveis[matName][2]:
                assm = ""
                j = 0
                for i in p[7]:
                    assm += f'''PUSHGP\nPUSHI {parser.variaveis[matName][0]}\nPADD\n{p[4]}PUSHI {parser.variaveis[matName][2]}\nMUL\nPADD\nPUSHI {j}\nPUSHI {i}\nSTOREN\n'''
                    j += 1
                p[0] = f'{assm}'
            else:
                parser.error = f"Tamanho da lista maior do que o alocado"
                parser.exito = False
        else:
            parser.error = f"Operação inválida, variável {matName} não é uma matriz"
            parser.exito = False
    else:
        parser.error = f"Variável não declarada anteriormente"
        parser.exito = False
    parser.linhaDeCodigo +=1


# Função que vai buscar o valor do indice na lista
def p_AtribBusca_Lista(p):
    "expr : BUSCA ID ABREPR expr FECHAPR"
    varName = p[2]
    indice = p[4]
    if varName in parser.variaveis:
        p[0] = f"PUSHGP\nPUSHI {parser.variaveis[varName][0]}\nPADD\n{indice}LOADN\n"
    else:
        parser.error = (
            f"Variável com o nome {varName} não definida anteriormente.")
        parser.exito = False
    parser.linhaDeCodigo +=1


# Função que vai buscar o valor do indice na matriz
def p_AtribBusca_Matriz(p):
    "expr : BUSCA ID ABREPR expr FECHAPR ABREPR expr FECHAPR"
    varName = p[2]
    indice1 = p[4]
    indice2 = p[7]
    if varName in parser.variaveis:
        p[0] = f"PUSHGP\nPUSHI {parser.variaveis[varName][0]}\nPADD\n{indice1}PUSHI {parser.variaveis[varName][2]}\nMUL\nPADD\n{indice2}LOADN\n"
    else:
        parser.error = f"Variável com o nome {varName} não definida"
        parser.exito = False
    parser.linhaDeCodigo +=1


# Função swap entre elementos do mesmo array
def p_ProcSwap_Lista(p):
    "Proc : SWAP ID ABREPR INT FECHAPR COM ABREPR INT FECHAPR"
    varName = p[2]
    indice1 = p[4]
    indice2 = p[8]
    if varName in parser.variaveis:
        p[0] = f"PUSHG {indice1}\nPUSHG {indice2}\nSTOREG {indice1}\nSTOREG {indice2}\n"
    else:
        parser.error = (
            f"Variável com o nome {varName} não definida anteriormente.")
        parser.exito = False
    parser.linhaDeCodigo +=1


# Expressão Aritmética Soma
def p_soma(p):
    "exprArit : expr SOMA expr"
    p[0] = f"{p[1]}{p[3]}ADD\n"


# Expressão Aritmética Subtração
def p_sub(p):
    "exprArit : expr MENUS expr"
    p[0] = f"{p[1]}{p[3]}SUB\n"


# Expressão Aritmética Multiplicação
def p_mult(p):
    "exprArit : expr SOMANBEZES expr"
    p[0] = f"{p[1]}{p[3]}MUL\n"


# Expressão Aritmética Divisão
def p_div(p):
    "exprArit : expr DIBIDE expr"
    p[0] = f"{p[1]}{p[3]}MUL\n"


# Expressão Aritmética Resto da divisão
def p_rem(p):
    "exprArit : expr SOBRAS expr"
    p[0] = f"{p[1]}{p[3]}MOD\n"


# Expressão Relativa Não
def p_not(p):
    "exprRel : NOUM ABREPC expr FECHAPC"
    p[0] = f"{p[3]}NOT\n"


# Expressão Relativa Igual
def p_gemeo(p):
    "exprRel : GEMEO ABREPC expr VIRG expr FECHAPC"
    p[0] = f"{p[3]}{p[5]}EQUAL\n"


# Expressão Relativa Diferente
def p_naogemeo(p):
    "exprRel : NAOGEMEO ABREPC expr VIRG expr FECHAPC"
    p[0] = f"{p[3]}{p[5]}NOT\nEQUAL\n"


# Expressão Relativa Menor
def p_inf(p):
    "exprRel : MAISPIQUENO ABREPC expr VIRG expr FECHAPC"
    p[0] = f"{p[3]}{p[5]}INF\n"


# Expressão Relativa Menor ou Igual
def p_infeq(p):
    "exprRel : MAISPIQUENOOUGEMEO ABREPC expr VIRG expr FECHAPC"
    p[0] = f"{p[3]}{p[5]}INFEQ\n"


# Expressão Relativa Maior
def p_sup(p):
    "exprRel : MAISGRANDE ABREPC expr VIRG expr FECHAPC"
    p[0] = f"{p[3]}{p[5]}SUP\n"


# Expressão Relativa Maior ou Igual
def p_supeq(p):
    "exprRel : MAISGRANDEOUGEMEO ABREPC expr VIRG expr FECHAPC"
    p[0] = f"{p[3]}{p[5]}SUPEQ\n"


# Expressão Relativa E
def p_ie(p):
    "exprRel : expr IE expr"
    p[0] = f"{p[1]}{p[3]}ADD\nPUSHI 2\nEQUAL\n"


# Expressão Relativa OU
def p_oue(p):
    "exprRel : expr OUE expr"
    p[0] = f"{p[1]}{p[3]}ADD\nPUSHI 1\nSUPEQ\n"


# Controlo de fluxo (if then)
def p_if_Then(p):
    "if : SE ABREPC exprRel FECHAPC ENTAO ABRECHAV Codigo FECHACHAV FIM"
    p[0] = f"{p[3]}JZ l{parser.labels}\n{p[7]}l{parser.labels}: NOP\n"
    parser.labels += 1
    parser.linhaDeCodigo+=1


# Controlo de fluxo (if then else)
def p_if_Then_Else(p):
    "if : SE ABREPC exprRel FECHAPC ENTAO ABRECHAV Codigo FECHACHAV SENAO ABRECHAV Codigo FECHACHAV FIM"
    p[0] = f"{p[3]}JZ l{parser.labels}\n{p[7]}JUMP l{parser.labels}f\nl{parser.labels}: NOP\n{p[11]}l{parser.labels}f: NOP\n"
    parser.labels += 1
    parser.linhaDeCodigo+=1


# Ciclo (while)
def p_while(p):
    "while : ENQUANTO ABREPC exprRel FECHAPC FAZ ABRECHAV Codigo FECHACHAV FIM"
    p[0] = f'l{parser.labels}c: NOP\n{p[3]}JZ l{parser.labels}f\n{p[7]}JUMP l{parser.labels}c\nl{parser.labels}f: NOP\n'
    parser.labels += 1
    parser.linhaDeCodigo+=1


def p_saidas_STRING(p):
    '''saidas : SAIDAS ASPA'''
    p[0] = f'PUSHS {p[2]}\nWRITES\n'
    parser.linhaDeCodigo+=1


def p_saidas_lista(p):
    "saidas : SAIDAS ID"
    if len(parser.variaveis[p[2]]) == 3:
        listas = parser.variaveis[p[2]]
        initLista = listas[0]
        numeroListas = listas[1]
        tamanhoListas = listas[2]
        assm = "PUSHS \"[\"\nWRITES\n"
        for i in range(numeroListas):
            assm += "PUSHS \"[\"\nWRITES\n"
            for j in range(tamanhoListas):
                assm += f"PUSHGP\nPUSHI {initLista}\nPADD\nPUSHGP\nPUSHI {i}\nPUSHI {tamanhoListas}\nMUL\nPADD\nPUSHI {j}\nLOADN\nWRITEI\nPOP 1\nPUSHS \",\"\nWRITES\n"
            rm = "PUSHS \",\"\nWRITES"
            assm = assm[:-len(rm)-1]
            assm += "PUSHS \"]\"\nWRITES\n"
            assm += "PUSHS \",\"\nWRITES\n"
        rm = "PUSHS \",\"\nWRITES"
        assm = assm[:-len(rm)-1]
        assm += "PUSHS \"]\"\nWRITES\n"
        p[0] = assm

    elif len(parser.variaveis[p[2]]) == 2:
        if parser.variaveis[p[2]][1] != None:
            listas = parser.variaveis[p[2]]
            initLista = listas[0]
            tamanhoListas = listas[1]
            assm = "PUSHS \"[\"\nWRITES\n"
            for j in range(tamanhoListas):
                assm += f"PUSHGP\nPUSHI {initLista}\nPADD\nPUSHI {j}\nLOADN\nWRITEI\nPUSHS \",\"\nWRITES\n"
            rm = "PUSHS \",\"\nWRITES"
            assm = assm[:-len(rm)-1]
            assm += "PUSHS \"]\"\nWRITES\n"
            p[0] = assm
        else:
            p[0] = f"PUSHG {parser.variaveis[p[2]][0]}\nWRITEI\n"

    else:
        parser.error = ""
        parser.exito = False
    parser.linhaDeCodigo+=1


# Funções auxiliares

def p_lista(p):
    "lista : ABREPR elems FECHAPR"
    p[0] = p[2]


def p_elems(p):
    "elems : INT"
    p[0] = [int(p[1])]


def p_elems_rec(p):
    "elems : elems VIRG INT"
    p[0] = p[1]+[p[3]]


# ----------------------------------------

def p_error(p):
    print(p)
    try:
        helper(p.value)
        parser.exito = False
    except:
        print(p)
        parser.exito=False


def helper(syntaxError):
    error = syntaxError.upper()
    matches = difflib.get_close_matches(error, tokens, n=2, cutoff=0.6)
    if matches != []:
        parser.error = f"Syntax error na linha {parser.linhaDeCodigo}: Querias dizer {matches[0]}"

# ----------------------------------------


parser = yacc.yacc()
parser.exito = True
parser.error = ""
parser.assembly = ""
parser.variaveis = {}
parser.stackPointer = 0
parser.linhaDeCodigo = 0
parser.labels = 0

assembly = ""


if len(sys.argv) == 3:
    inputFileName = sys.argv[1]
    if inputFileName[-4:] == ".plo":
        file = open(inputFileName, "r")
        content = file.read()
        parser.parse(content)
        if parser.exito:
            assembly += parser.assembly
        else:
            print("--------------------------------------")
            print(parser.error)
            print("--------------------------------------")
            sys.exit()
        file.close()

        arr = os.listdir()
        outputFileName = sys.argv[2]

        while outputFileName in arr:
            outputFileName = outputFileName.split(".")[0]
            outputFileName += "_copy.vm"

        if ".vm" not in outputFileName:
            outputFileName+=".vm"

        outputFile = open(outputFileName, "w")
        outputFile.write(assembly)
        outputFile.close()

        print("File saved successfully")

    else:
        print("Invalid file extension")


if len(sys.argv) == 2:
    inputFileName = sys.argv[1]
    if inputFileName[-4:] == ".plo":
        file = open(inputFileName, "r")
        content = file.read()
        parser.parse(content)
        if parser.exito:
            assembly += parser.assembly
            print(parser.variaveis)
        else:
            print("--------------------------------------")
            print(parser.error)
            print(parser.variaveis)
            print("--------------------------------------")
            sys.exit()
        file.close()
        outputFileName = "a.vm"

        arr = os.listdir()

        while outputFileName in arr:
            outputFileName = outputFileName.split(".")[0]
            outputFileName += "_copy.vm"

        outputFile = open(outputFileName, "w")
        outputFile.write(assembly)
        outputFile.close()

        print("File saved successfully")

    else:
        print("Invalid file extension")

if len(sys.argv) == 1:
    line = input(">")
    while line:
        parser.exito = True
        parser.parse(line)
        if parser.exito:
            assembly += parser.assembly
        else:
            print("--------------------------------------")
            print(parser.error)
            print("--------------------------------------")
            sys.exit()
        line = input(">")

    saveMachineCode = input(
        "Do you want to save the code that you generated?[y/n]")
    if saveMachineCode.lower() == "y":
        path = input("Where do you want to save it?")
        if path:
            if ".vm" not in path:
                file = open(f"{path}.vm", "w")
                file.write(assembly)
            else:
                file = open(f"{path}.vm", "w")
                file.write(assembly)

        else:
            file = open("./a.vm", "w")
            file.write(assembly)

        file.close()
        print("File saved successfully")

    else:
        print("Bye Bye")
        quit

\end{lstlisting}
\end{scriptsize}

\textbf{Ficheiro de teste factorial.plo}
\begin{scriptsize}
\begin{lstlisting}[language=python]
saidas "Factorial: "
var n com entradas
saidas n
var res com 1

enquanto (maisGrande(n,0)) faz {
    alterna res com res * n
    alterna n com n - 1
} fim

saidas "\nResultado: "
saidas res
\end{lstlisting}
\end{scriptsize}


\textbf{Ficheiro de teste busca\_no\_array.plo}
\begin{scriptsize}
\begin{lstlisting}[language=python]
lista a 10
lista a com [1,2,3,4,5,6,7,8,9,10]

saidas "Introduza um indice do array:\n"
var i com entradas

var x com busca a[i]

saidas "Valor: "
saidas x
\end{lstlisting}
\end{scriptsize}

\textbf{Ficheiro de teste\ read\_array.plo}
\begin{scriptsize}
\begin{lstlisting}[language=python]
var n com 5
var i com 0
lista a 5

enquanto (maisPiqueno(i,n)) faz {
    alterna a [i] com entradas
    alterna i com i + 1
} fim

saidas "Array gerado:\n"
saidas a
\end{lstlisting}
\end{scriptsize}


\textbf{Ficheiro de teste\ produtorio.plo}
\begin{scriptsize}
\begin{lstlisting}[language=python]
saidas "Quantos numeros? "
var n com entradas
saidas n
var res com 1
var x com 1

enquanto (maisGrande(n,0)) faz {
    alterna x com entradas
    alterna res com res * x
    alterna n com n - 1
} fim

saidas "\nResultado: "
saidas res
\end{lstlisting}
\end{scriptsize}

\textbf{Ficheiro de teste\ matriz.plo}
\begin{scriptsize}
\begin{lstlisting}[language=python]
matriz m 3 3

alterna m [0] com [1,2,3]
alterna m [1] com [4,5,6]
alterna m [2] com [7,8,9]

saidas m
\end{lstlisting}
\end{scriptsize}

\textbf{Ficheiro de teste\ swap\_array.plo}
\begin{scriptsize}
\begin{lstlisting}[language=python]
lista a 5
lista a com [1,2,3,4,5]

saidas "Array inicial:\n"
saidas a

saidas "\nTroca do indice 1 com indice 3."

swap a [1] com [3]

saidas "Array inicial:\n"
saidas a
\end{lstlisting}
\end{scriptsize}



\end{document} 